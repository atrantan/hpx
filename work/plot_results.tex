\documentclass[svgnames]{beamer}
\usepackage{listings}
\usepackage[utf8x]{inputenc}
\usepackage{color}
\usepackage{xcolor}
\usepackage[frenchb,english]{babel}
\usepackage{standalone}
\usepackage{tikz}
\usepackage{pgfplots}

\pgfplotsset{compat=1.11}

\definecolor{metared}{rgb}{0.64,0.13,0.11}
\definecolor{hlcolor}{rgb}{0.88,0.88,0.88}

\definecolor{dkgreen}{rgb}{0,0.6,0}
\definecolor{gray}{rgb}{0.5,0.5,0.5}
\definecolor{mauve}{rgb}{0.58,0,0.82}

\lstset{ %
  language=C++,                % the language of the code
  basicstyle=\ttfamily\tiny,           % the size of the fonts that are used for the code
  showspaces=false,               % show spaces adding particular underscores
  showstringspaces=false,         % underline spaces within strings
  showtabs=false,                 % show tabs within strings adding particular underscores
  frame=none,                     % adds a frame around the code
  tabsize=2,                      % sets default tabsize to 2 spaces
  captionpos=b,                   % sets the caption-position to bottom
  breaklines=true,                % sets automatic line breaking
  breakatwhitespace=false,        % sets if automatic breaks should only happen at whitespace
  xleftmargin=\fboxsep,
  xrightmargin=-\fboxsep,
  firstnumber=1,
  title=\lstname,                   % show the filename of files included with \lstinputlisting;
  keywordstyle=\color{blue},          % keyword style
  commentstyle=\color{dkgreen},       % comment style
  % stringstyle=\color{mauve},         % string literal style
  escapeinside={\%*}{*)},            % if you want to add a comment within your code
  morekeywords={*,...},               % if you want to add more keywords to the set
  escapechar={@},
  aboveskip=0bp,
  belowskip=0bp
}

\title{Extending C++ with Co-Array semantics}
\author{Antoine Tran Tan \and Hartmut Kaiser}
\institute{Louisiana State University\\Center for Computation and Technology - Stellar Group}
\date{\footnotesize{May 2016}}

\begin{document}

%=============================================
\begin{frame}
\titlepage
\end{frame}
%=============================================
\begin{frame}
\frametitle{Matrix Transpose - Performance evaluation}
\begin{center}
\begin{tikzpicture}[scale = 0.9]
\pgfplotsset{grid style={dotted,gray}}
\begin{axis}[xlabel=\footnotesize Number of MPI processes,
            ylabel=\footnotesize Performance in Gb/s,grid,
            legend pos=north west]
  \addplot[color=black]
  plot[smooth,mark=diamond*,mark size=1.5pt] file
  {perfs/pvector_view_transpose/pvector_view_transpose_matrix_order=2000_partition_order=200_scale_over_nprocs.dat};
  \addlegendentry{\scriptsize{matrix order = 2000, partition order = 200}}
  \addplot[color=blue]
  plot[smooth,mark=diamond*,mark size=1.5pt] file
  {perfs/pvector_view_transpose/pvector_view_transpose_matrix_order=4000_partition_order=400_scale_over_nprocs.dat};
  \addlegendentry{\scriptsize{matrix order = 4000, partition order = 400}}
  \addplot[color=red]
  plot[smooth,mark=diamond*,mark size=1.5pt] file
  {perfs/pvector_view_transpose/pvector_view_transpose_matrix_order=6000_partition_order=600_scale_over_nprocs.dat};
  \addlegendentry{\scriptsize{matrix order = 6000, partition order = 600}}
  \addplot[color=red]
  plot[smooth,mark=diamond*,mark size=1.5pt] file
  {perfs/pvector_view_transpose/pvector_view_transpose_matrix_order=8000_partition_order=800_scale_over_nprocs.dat};
  \addlegendentry{\scriptsize{matrix order = 8000, partition order = 800}}
  \addplot[color=dkgreen]
  plot[smooth,mark=diamond*,mark size=1.5pt] file
  {perfs/pvector_view_transpose/pvector_view_transpose_matrix_order=10000_partition_order=1000_scale_over_nprocs.dat};
  \addlegendentry{\scriptsize{matrix order = 10000, partition order = 1000}}
\end{axis}
\end{tikzpicture}
\end{center}
\end{frame}
%=============================================
\end{document}
